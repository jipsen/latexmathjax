\documentclass{amsart} % this is ignored

%\usepackage{tikz} % this is ignored

\newtheorem{theorem}{Theorem} % these are ignored
\newtheorem{lemma}[theorem]{Lemma}
\newtheorem{corollary}[theorem]{Corollary}

\title[short title]{Testing LaTeXMathJax} % short title is ignored

\author{Author Name}
\email{author@email.edu}
\address{Department of Mathematics\\ % \\ are replaced by comma
University of LaTeX\\
1234 MathJax St.\\
City, ST 12345, Country}

\author{Second Author}
\email{author2@email.edu}
\address{HTML University\\
Faculty of CSS\\
One University Drive\\
City, ST 12345, Country}

\date{February 25, 2014}

\keywords{keyword1, keyword2}

%(re)newcommands are ignored
\renewcommand{\subjclassname}{\textup{2000} Mathematics Subject Classification}
\subjclass{Primary: xxAyy, Secondary: xxByy, xxCyy}

\begin{document} % this is ignored

\begin{abstract}
This is a test article to check if all the regex translation rules
are applied correctly.
\end{abstract}

\maketitle % this is ignored

\section{Introduction}
Some famous person~\cite{FP99} defined \emph{some important objects} as 
a special case of the following more general stuff:
$$
\text{a displayed equation: }\sqrt{r^2-x^2}.
$$

\begin{center}
Several lines that

should be centered
\end{center}

\begin{figure}
%\begin{tikzpicture}
%This stuff is deleted since tikz is not handled (yet)
%\end{tikzpicture}
\caption{Here is the figure caption}
\end{figure}

Theorem~\ref{C} below is proved with the help of several lemmas.

\begin{lemma}\label{A} Some fairly easy result.
\end{lemma}
\begin{proof}
It is true because of the following simple steps:

Step1

Step2

Done.
\end{proof}

\section{The main result} 
\subsection{To show that subsection numbering works}
\subsection{And another subsection}
\subsubsection{With a subsubsection}
Before we can prove this result, we need one
more lemma.

\begin{lemma}\label{B}
\begin{enumerate}
\renewcommand{\theenumi}{\roman{enumi}} % this is ignored
  \item First item
  \item Second item
\end{enumerate}
\end{lemma}

\begin{proof}
(1) is obvious.

(2) needs some careful thought.
\end{proof}

\begin{theorem}\label{C} This is the main result.
\end{theorem}
\begin{proof}
Follows from the preceeding lemmas.
\end{proof}
Note that the converse of Lemma~\ref{A} also holds.

\begin{corollary}
This result is a simple consequence of the previous theorem.
\end{corollary}

\begin{thebibliography}{}
\bibitem{FP99} Famous Person,
\emph{A wonderful result}, 
J. of Math. 123 (1999), no. 1, 1--23. 
\end{thebibliography}
\end{document}
